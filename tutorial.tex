\documentclass[12pt,journal,compsoc]{IEEEtran}
\usepackage{graphicx}
\usepackage{hyperref}
\begin{document}

\title{LaTeX Tutorial}
\author{Ankit Sachdeva}

\date{\today}		% leaving the brackets empty omits the date

\markboth{\LaTeX\ Tutorial}%
{Moulds \MakeLowercase{\textit{et al.}}: CMPE185}

\IEEEpubid{\copyright~2022 Ankit Sachdeva}
\maketitle

\section{Introduction}

\IEEEPARstart {} \LaTeX \space is a typesetting program which was originally designed in the early 1990s for books but is often used today in academic settings to write papers, journals, and tutorials. \LaTeX \space allows almost infinite formatting options and can be used for pretty much anything. That being said, the learning curve for \LaTeX \space is high and just getting started can be time consuming. When writing something like a book or a research paper, this trade-off of initial setup is negligible and worth it for the ability to format documents exactly like you want. 

\subsection{Why Learn \LaTeX}
At some point either in your professional, academic, or personal life, you will need to format a document in a way that regular "What You See Is What You Get" text editor such as Microsoft Word or Google Docs won't let you do. Alternatively, you might be writing a paper for a journal that requires a specific format. 

\subsection{Why This Tutorial}
This tutorial will go over the fundamentals of \LaTeX \space and get you situated to write a basic document. It is by no means a complete tutorial and can be used as a 'quick reference' guide anytime that you find yourself writing a document in \LaTeX \space . 

\subsection{Getting Started}
Getting started with \LaTeX \space can be a difficult. There are options like \href{https://www.overleaf.com/}{Overleaf} that allow you to edit and compile \LaTeX \space directly from your browser. Personally, I do not like these sorts of solutions as they paywall key features such as version control and all your work is dependent on their servers being online. Instead, I strongly suggest downloading \LaTeX \space locally on your computer and creating a Makefile to compile your files into a PDF. A simple installation script and Makefile template on how to do this can be found \href{https://gist.github.com/ankitsxchdeva/1eb2afaad782a92b9a129a9452ae97bb}{here} but are outside the scope of this tutorial. A text editor such as vim or nano can be combined with \LaTeX \space allowing easier navigation of large files. 


\section{Creating a .tex file}
There are a lot of filetypes when compiling a \LaTeX \space document. However, there are only 2 filetypes that we have to concern ourselves with when writing basic documents: .tex (LaTeX Source Document) and .cls (LaTeX Document Class). You can think of the .cls file like a template file. For example this file is being compiled with the IEEEtran.cls file. This tutorial will focus on editing .tex files where all of the content and structure of a document is defined.

\subsection{Environments}
Environments in \LaTeX \space are used to format sections of text in specific ways. An environment called \verb|xyz| would start with \verb|\begin{xyz}| and end in \verb|\end{xyz}| with the special environment rules applying to all the text inbetween those tags. There are lots of built-in environments such as \verb|center| which centers all the text or \verb|verbatim| which ignores any reserved characters and prints the texts in monospace font. All documents start and end in the document environment.

\subsection{Reserved Characters}
There are some reserved characters in \LaTeX \space that introduce some complexity when writing in \LaTeX \space that need to be considered 

\section{Sections}
Creating sections and subsections in \LaTeX \space is super easy and 



\section{Additional Features}

\subsection{Figures}
% FIGURES:

% Note the FIGURE Environment created by the \begin{figure} and \end{figure} commands.

\begin{figure}[h] 	% There are several different modifiers that can be used in [].
\centering
\includegraphics[width=1.5in]{slug.pdf}
\caption{Sammy the Slug}
\label{fig_slug}
\end{figure}

% You will need to use appropriate file types for figures and will also need to include that image file in the same folder as your .tex file. 

%-------------------------------------------------------------------------------------------------
\subsection{Label, Cite, and Ref Commands}
You will need to be able to define and explain how to use the following commands:
\begin{verbatim} \label{fig_slug} \end{verbatim} as it appears in the figure environment\\
\begin{verbatim} \cite{IEEEhowto:kopka} \end{verbatim} appears like: \cite{IEEEhowto:kopka}\\
\begin{verbatim} \ref{fig_slug} \end{verbatim} appears like: \ref{fig_slug}\\

% IMPORTANT NOTE: In order to assign the correct reference number to each label, you may have to compile your code twice. 

%-------------------------------------------------------------------------------------------------
\subsection{Tables}
An example of a floating table.

% An example of a floating table. Note that, for IEEE style tables, the 
% \caption command should come BEFORE the table. Table text will default to
% \footnotesize as IEEE normally uses this smaller font for tables.
% The \label must come after \caption as always.
%
\begin{table}[h]
%% increase table row spacing, adjust to taste
\renewcommand{\arraystretch}{1.3}
% if using array.sty, it might be a good idea to tweak the value of
%\extrarowheight as needed to properly center the text within the cells
\caption{An Example of a Table}
\label{table_example}
\centering
%% Some packages, such as MDW tools, offer better commands for making tables
%% than the plain LaTeX2e tabular which is used here.
\begin{tabular}{|c||c|}
\hline
One & Two\\
\hline
Three & Four\\
\hline
\end{tabular}
\end{table}


% Note that IEEE does not put floats in the very first column - or typically
% anywhere on the first page for that matter. Also, in-text middle ("here")
% positioning is not used. Most IEEE journals use top floats exclusively.
% However, Computer Society journals sometimes do use bottom floats - bear
% this in mind when choosing appropriate optional arguments for the
% figure/table environments.
% Note that, LaTeX2e, unlike IEEE journals, places footnotes above bottom
% floats. This can be corrected via the \fnbelowfloat command of the
% stfloats package.


\section{Conclusion}
Conclusion goes here.

%----- APPENDICES --------------------------------------------------------------------------------
\appendices
\section{Appendix Title}
Appendix one text goes here.

% you can choose not to have a title for an appendix
% if you want by leaving the argument blank
\section{}
Appendix two text goes here.


%----- ACKNOWLEDGEMENT SECTION -------------------------------------------------------------------
% Explain what the asterisk * does in the next line: 
\section*{Acknowledgements}

The author would like to thank...\\ \\

Reminder: you will need to explain how to include an Acknowledgement Section and then include your own Acknowledgement Section at the end of your own tutorial. Same applies for the References/Bibliography.


%----- BIBLIOGRAPHY ------------------------------------------------------------------------------

% You will need to explain how to include the bibliography section as follows. Explain the environment and how to add new items.
% Including how \ref, \cite and \label should be included here.

% Reminder: you will need to explain how to include the Bibliography Section and then include your own Bibliography at the end of your own tutorial.

\begin{thebibliography}{1}

\bibitem{IEEEhowto:kopka}
H.~Kopka and P.~W. Daly, \emph{A Guide to {\LaTeX}}, 3rd~ed.\hskip 1em plus
  0.5em minus 0.4em\relax Harlow, England: Addison-Wesley, 1999.

\end{thebibliography}

%----- Optional: BIOGRAPHY Section ---------------------------------------------------------------
 
% If you have an EPS/PDF photo (graphicx package needed) extra braces are
% needed around the contents of the optional argument to biography to prevent
% the LaTeX parser from getting confused when it sees the complicated
% \includegraphics command within an optional argument. (You could create
% your own custom macro containing the \includegraphics command to make things
% simpler here.)
%\begin{biography}[{\includegraphics[width=1in,height=1.25in,clip,keepaspectratio]{mshell}}]{Gerald Moulds}
% or if you just want to reserve a space for a photo:

\begin{IEEEbiography}{Gerald Moulds}
Biography text here.
\end{IEEEbiography}

% if you will not have a photo at all:
\begin{IEEEbiographynophoto}{John Doe}
Biography text here.
\end{IEEEbiographynophoto}

% insert where needed to balance the two columns on the last page with
% biographies
%\newpage

\begin{IEEEbiographynophoto}{Jane Doe}
Biography text here.
\end{IEEEbiographynophoto}

% You can push biographies down or up by placing
% a \vfill before or after them. The appropriate
% use of \vfill depends on what kind of text is
% on the last page and whether or not the columns
% are being equalized.

%\vfill

% Can be used to pull up biographies so that the bottom of the last one
% is flush with the other column.
%\enlargethispage{-5in}

\end{document}



% https://www.elsevier.com/authors/policies-and-guidelines/latex-instructions
% https://www.latex-project.org/about/
% https://github.com/asilva3/CMPE185/blob/master/LaTeX.pdf
% https://stackoverflow.com/questions/21577968/how-to-tell-if-homebrew-is-installed-on-mac-os-x
% https://www.latex-project.org/get/
%https://www.overleaf.com/learn/latex/Environments
% https://www.overleaf.com/learn/latex/Code_listing
